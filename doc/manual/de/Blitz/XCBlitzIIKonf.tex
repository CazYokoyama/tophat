\documentclass[german,a4paper,12pt,utf8]{scrreprt}
\usepackage[utf8]{inputenc}
\usepackage[german]{babel}
%\usepackage[T1]{fontenc}
\usepackage[off]{flippdf}                %Spiegelverkehrte Ausgabe ganzer Seiten (sogar Bilder ..) -> Default OFF ..
\usepackage{supertabular}
\usepackage{array}
\usepackage{wasysym}
\usepackage{hhline}
\usepackage{framed}               % Farbunterstützung
\usepackage{color}               % Farbunterstützung
\usepackage{graphicx}                      % Anleitung hierzu  in c:\Program Files\MiKTeX 2.7\doc\latex\graphics\grfguide.pdf
\usepackage{wrapfig}
\usepackage{fancyhdr}
\usepackage{fancybox}
\usepackage{booktabs}
\usepackage{paralist}
\usepackage{fancyhdr}
\usepackage{url}
% Wir wollen einen Index:
\usepackage{makeidx}\makeindex
% Wir wollen aktive Links und einige Dokumentinformationen:
\usepackage[colorlinks]{hyperref}
\hypersetup{
    bookmarks=true,
    unicode=true,
    pagebackref=true,
    pdftoolbar=true,
    pdfmenubar=true,
    pdffitwindow=false,
    pdfstartview={FitH},
    pdftitle={XCSoar Blitzeinstieg},
    pdfauthor={OH},
    pdfsubject={Subject},
    pdfcreator={HotteTeX},
    pdfproducer={OH},
    pdfkeywords={XCSoar}{Schnelleinstieg}{Segelflug}{Endanflug},
    pdfnewwindow=true,
    colorlinks=true,
    linkcolor=red,
    citecolor=green,
    filecolor=magenta,
    urlcolor=blue
}
\usepackage[right]{eurosym}% Eurosymbol, rechts neben Text mit \EUR{427,45}, symbol alleine mit\eur{}
\parindent0mm%kein Einzug-häßlich
\textwidth=170mm
\rightmargin=0mm\leftmargin=-10mm\oddsidemargin=-10mm\evensidemargin=-10mm
\setlength{\baselineskip}{1em}
% Font selection
\usepackage{helvet}
\renewcommand{\familydefault}{\sfdefault}
\fontfamily{phv}\selectfont
%
%##############  Farben, Hintergründe, Rahmen ##############
%
\fboxrule0.4mm                                                      % Breite des Fbox-Rahmens
%
\definecolor{gray}{rgb}{0.8,0.8,0.8}                       %{0.7,0.7,0.7}
\definecolor{buttongreen}{rgb}{0.625,0.94,0.625}  %
\definecolor{rahmen1}{gray}{0.5}                            % hellgrauer Rahmen - entweder RGB -Angabe oder wie hier
\definecolor{hintergrund1}{rgb}{.7,1,.7}                 % hellgrüner Hintergund
\definecolor{rahmen2}{gray}{0.1}                            % grau eben
\definecolor{hintergrund2}{rgb}{.9,.9,.9}               % ganz hellgrauer Hintergrund
\definecolor{rahmen3}{gray}{0.15}                         % grau eben
\definecolor{hintergrund3}{rgb}{1.0,1.0,0.9}         %hellgelber Hintergund
% \definecolor{hintergrund3}{rgb}{.7,1,1}              % ganz türkis Hintergrund
%
%##############  Makros und  Menus ..##############
%
\newcommand{\bc}{\begin{center}}
\newcommand{\ec}{\end{center}}
\newcommand{\blink}[0]{{~\LARGE$\triangleright$~}}
%
\newcommand{\bmg}[1]{\fcolorbox {black}{gray}{{\strut\sf{\footnotesize#1}}}}
%
\newcommand{\bmw}[1]{\fcolorbox {black}{white}{{\strut\sf{\footnotesize#1}}}}
%
\newcommand{\bmt}[2]{%Menübutton, zweizeilig
\fcolorbox {black}{gray}{
    \makebox[1.6cm][c]{
    \begin{tabular}{c}
    {\footnotesize\sf{#1}}\\
    {\footnotesize\sf{#2}}
    \end{tabular}
    }
  }
}
%
\newcommand{\bms}[1]{%Menübutton, single, einzeilig
\fcolorbox {black}{gray}{
    \makebox[1.6cm][c]{
      \begin{tabular}{c}
        {\footnotesize\sf{#1}}\\
      \\
    \end{tabular}
  }
}
}
%
\newcommand{\button}[1]{% einfache Box für Schaltfläche/Button
\fcolorbox {black}{gray}{{\sf #1}}}
%
%##############  Erweiterte Makros und  Menus ..##############
\newcommand{\nav}[3]{\bmt{Nav}{#1/2}{\LARGE$\triangleright$}~\bmt{#2}{#3}}%NavMenu
\newcommand{\ansi}[3]{\bmt{Ansicht}{#1/2}{\LARGE$\triangleright$}~\bmt{#2}{#3}}%AnsichtMenu
\newcommand{\konf}[3]{\bmt{Konfig}{#1/3}{\LARGE$\triangleright$}~\bmt{#2}{#3}}%KonfigNavMenu
\newcommand{\info}[3]{\bmt{Info}{#1/3}{\LARGE$\triangleright$}~\bmt{#2}{#3}}%InfoMenu

\newcommand{\sk}[0]{%  Häufig benutzt .. konf-konf-system-einstellungen
\fcolorbox {black}{gray}{
    \makebox[1.6cm][c]{
    \begin{tabular}{c}
    {\footnotesize\sf{Konfig}}\\
    {\footnotesize\sf{2/3}}
    \end{tabular} } }
{\LARGE$\triangleright$}\hspace{0.0075em}
\fcolorbox {black}{gray}{
    \makebox[1.6cm][c]{
    \begin{tabular}{c}
    {\footnotesize\sf{System}}\\
    {\footnotesize\sf{Einstellung}}
    \end{tabular} } } }
%Abkürzungen
\newcommand{\xc}{{\textsf XCSoar}}
%
%##############  Titelseite des Blitzes .##############
\def\maketitle{%
  \null
  \thispagestyle{empty}%
  \begin{maxipage}
    \begin{center}
     \includegraphics[angle=0,width=0.5\textwidth,keepaspectratio='true']{figures/xcsoar-title.png}
    \vskip 0.5cm

    \end{center}
    \begin{center}
      \normalfont\huge\textsf{der \textbf{Blitzeinstieg} zu dem quelloffenen-Segelflugrechner für fast jede Hardware}\par
    \end{center}
    \vskip 1cm
    \begin{center}
      \normalfont\huge\textsf{\@title}\par
    \end{center}
    \vskip 1cm
  \end{maxipage}

  \vfill
  %\todo[nolist,size=\Large,inline]{Beta-Stadium, inhaltlich OK, am Outfit gibt es noch zu tun\dots}

  \begin{flushright}
    \large \strut {
      \sf
      \today \\
      Für \xc Version \version \\
      \xcsoarwebsite \\
    }
    \par
  \end{flushright}
  \par
  \vfil
  \vfil
  \null
  \cleardoublepage
}
%
