\chapter{Compiling XCSoar}\label{cha:compiling}

The \texttt{make} command is used to launch the XCSoar build process.
You can learn more about the build system internals in chapter
\ref{cha:build}.

Most of this chapter describes how to build XCSoar on Linux, with
examples for Debian/Ubuntu.  A cross-compiler is used to build
binaries for other operating systems (for example Android and
Windows).

\section{Getting the Source Code}

The XCSoar source code is managed with
\href{http://git-scm.com/}{git}.  It can be downloaded with the
following command:

\begin{verbatim*}
git clone git://git.xcsoar.org/xcsoar/master/xcsoar.git
\end{verbatim*}

To update your repository, type:

\begin{verbatim*}
git pull
\end{verbatim*}

For more information, please read to the git documentation.

\section{Requirements}

The following is needed for all targets:

\begin{itemize}
\item GNU make
\item GNU compiler collection (\texttt{gcc}), version 4.6.2 or later
  or clang/LLVM 3.1 (with "make CLANG=y")
\item GNU gettext
\item \href{http://librsvg.sourceforge.net/)}{rsvg}
\item \href{http://www.imagemagick.org/}{ImageMagick 6.4}
\item \href{http://xmlsoft.org/XSLT/xsltproc2.html}{xsltproc}
\item Perl and XML::Parser
\end{itemize}

The following command installs these on Debian:

\begin{verbatim*}
apt-get install make \
  librsvg2-bin xsltproc \
  imagemagick gettext
\end{verbatim*}

\section{Target-specific Build Instructions}

\subsection{Compiling for Linux/UNIX}

The following additional packages are needed to build for Linux and
similar operating systems:

\begin{itemize}
\item \href{http://www.boost.org/}{Boost}
\item \href{http://www.zlib.net/}{zlib}
\item \href{http://curl.haxx.se/}{CURL}
\item \href{http://www.libsdl.org/}{SDL}
\item \href{http://www.libsdl.org/projects/SDL\_ttf/}{SDL\_ttf}
\item \href{http://www.libpng.org/}{libpng}
\item \href{http://libjpeg.sourceforge.net/}{libjpeg}
\item OpenGL (Mesa)
\item to run XCSoar, you need one of the following fonts (Debian
  package): DejaVu (\texttt{ttf-dejavu}), Droid (\texttt{ttf-droid}),
  Freefont (\texttt{ttf-freefont})
\end{itemize}

The following command installs these on Debian:

\begin{verbatim*}
apt-get install make g++ \
  libboost-dev zlib1g-dev \
  libsdl1.2-dev libsdl-ttf2.0-dev \
  libpng-dev libjpeg-dev \
  libcurl4-openssl-dev \
  libxml-parser-perl \
  librsvg2-bin xsltproc \
  imagemagick gettext \
  ttf-dejavu
\end{verbatim*}

To compile, run:

\begin{verbatim*}
make
\end{verbatim*}

You may specify one of the following targets with \texttt{TARGET=x}:

\begin{tabular}{lp{8cm}}

\texttt{UNIX} & regular build (the default setting) \\

\texttt{UNIX32} & generate 32 bit binary \\

\texttt{UNIX64} & generate 64 bit binary \\

\texttt{OPT} & alias for UNIX with optimisation and no debugging \\

\end{tabular}

\subsection{Compiling for Android}

For Android, you need:

\begin{itemize}
\item \href{http://developer.android.com/sdk/}{Android SDK level 16}
\item \href{http://developer.android.com/sdk/ndk/}{Android NDK r8e}
\item \href{http://www.vorbis.com/}{Ogg Vorbis}
\item \href{http://ant.apache.org/}{Apache Ant}
\end{itemize}

To compile, run:

\begin{verbatim*}
make TARGET=ANDROID
\end{verbatim*}

Use one of the following targets:

\begin{tabular}{lp{8cm}}

\texttt{ANDROID} & for ARMv6 CPUs \\

\texttt{ANDROID7} & for ARMv7 CPUs \\

\texttt{ANDROID7NEON} & with
\href{http://www.arm.com/products/processors/technologies/neon.php}{NEON}
extension \\

\texttt{ANDROID86} & for x86 CPUs \\

\texttt{ANDROIDMIPS} & for MIPS CPUs \\

\texttt{ANDROIDFAT} & "fat" package for all supported CPUs \\

\end{tabular}

\subsubsection{Enabling IOIO Support}

IOIO support must be enabled explicitly, because it depends on
IOIOLib.  First get the IOIOLib source code:

\begin{verbatim*}
git clone git://github.com/ytai/ioio.git
\end{verbatim*}

Now add the parameter \verb|IOIOLIB_DIR| pointing to this repository:

\begin{verbatim*}
make TARGET=ANDROID clean
make TARGET=ANDROID \
 IOIOLIB_DIR=/usr/src/git/ioio/software/IOIOLib
\end{verbatim*}

The first command may be necessary if the output directory already
contains binaries without IOIO support.

\subsection{Compiling for Windows}

To cross-compile to (desktop) Windows, you need the mingw-w64 version
of gcc:

 http://mingw-w64.sourceforge.net/

To compile, run one of the following:

\begin{verbatim*}
make TARGET=PC
\end{verbatim*}

Use one of the following targets:

\begin{tabular}{lp{8cm}}

\texttt{PC} & 32 bit Windows (i686) \\

\texttt{WIN64} & Windows x64 (amd64 / x86-64) \\

\texttt{WINE} & WineLib (experimental) \\

\texttt{CYGWIN} & Windows build with Cygwin (experimental) \\

\end{tabular}

\subsection{Compiling for Windows CE}

For PocketPC / Windows CE / Windows Mobile, you need mingw32ce:

\begin{itemize}
\item \href{http://max.kellermann.name/projects/cegcc/}{mingw32ce}
\end{itemize}

To compile, run:

\begin{verbatim*}
make TARGET=WM5X
\end{verbatim*}

Use one of the following targets:

\begin{tabular}{lp{8cm}}

\texttt{PPC2000} & PocketPC 2000 / Windows CE 3.0 \\

\texttt{PPC2003} & PocketPC 2003 / Windows CE 4.0 \\

\texttt{PPC2003X} & for XScale CPUs \\

\texttt{WM5} & Windows Mobile / Windows CE 5.0 \\

\texttt{WM5X} & for XScale CPUs \\

\texttt{ALTAIR} & for Triadis Altair \\

\end{tabular}

\subsection{Compiling for Mac OS X}

For Mac OS X, you need:

\begin{itemize}
\item GCC 4.6.2 or newer (http://hpc.sourceforge.net/, or homebrew, or Macports)
\item \href{http://www.boost.org/}{Boost}
\item \href{http://www.zlib.net/}{zlib}
\item \href{http://curl.haxx.se/}{CURL}
\item \href{http://www.libsdl.org/}{SDL}
\item \href{http://www.libsdl.org/projects/SDL\_ttf/}{SDL\_ttf}
\item \href{http://www.libpng.org/}{libpng}
\item \href{http://libjpeg.sourceforge.net/}{libjpeg}
\item \href{http://icns.sourceforge.net/}{libicns}
\end{itemize}

\subsection{Compiling for the Raspberry Pi}

To compile, run:

\begin{verbatim*}
make TARGET=PI
\end{verbatim*}

\subsection{Compiling for Kobo E-book Readers}

To compile, run:

\begin{verbatim*}
make TARGET=KOBO
\end{verbatim*}
